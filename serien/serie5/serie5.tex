\documentclass[12pt, a4paper]{article}
\usepackage{url,graphicx,tabularx,array,geometry}
\usepackage[utf8]{inputenc}
\usepackage[ngerman]{babel}
\usepackage{paralist}
\usepackage{latexsym}
\usepackage{fancyhdr}
\usepackage{siunitx}
\usepackage{graphicx}
\usepackage{float}
\usepackage{color}
\usepackage{pgf}
\usepackage{tikz}
\usetikzlibrary{arrows,automata}

\pagestyle{fancy}

\usepackage{amsmath}
\usepackage{amsfonts}
\usepackage{amssymb}

\setlength{\parskip}{1ex} %--skip lines between paragraphs
\setlength{\parindent}{0pt} %--don't indent paragraphs

%-- Commands for header
\newcommand{\headerline}{\begin{tabularx}{\textwidth}{X>{\raggedleft}X}\hline\\\end{tabularx}\\[-0.5cm]}
\newcommand{\headerleftright}[2]{\begin{tabularx}{\textwidth}{X>{\raggedleft}X}#1%
& #2\\\end{tabularx}\\[-0.5cm]}
%\linespread{2} %-- Uncomment for Double Space

\usepackage{listings}
\usepackage{color}

\definecolor{dkgreen}{rgb}{0,0.6,0}
\definecolor{gray}{rgb}{0.5,0.5,0.5}
\definecolor{mauve}{rgb}{0.58,0,0.82}

\lstset{frame=tb,
  language=Java,
  aboveskip=3mm,
  belowskip=3mm,
  showstringspaces=false,
  columns=flexible,
  basicstyle={\small\ttfamily},
  numbers=none,
  numberstyle=\tiny\color{gray},
  keywordstyle=\color{blue},
  commentstyle=\color{dkgreen},
  stringstyle=\color{mauve},
  breaklines=true,
  breakatwhitespace=true
  tabsize=3
}

\begin{document}
\renewcommand{\headrulewidth}{0pt}
\fancyhf{}
\fancyhead[L]{
\headerleftright{\textbf{TGI}}{David Elvers, Daniel Schmidt}}
\fancyfoot[C]{\thepage}

\section*{5.2}
@David: FILL ME!

\section*{5.3}
Um zu zeigen, dass eine Sprache kontextfrei ist zeigen wir, dass es eine Grammatik gibt die Kontextfrei ist und diese Sprache erzeugt.
Diese definieren wir wie folgt:

\begin{align*}
X^n &= \bigcup^{n \in \mathbb{N}_{\ge 0}} \{a^i b^j \mid \exists i,j \in \mathbb{N}_{\ge 0}: i + j = n \} \\
Y^n &= \bigcup^{n \in \mathbb{N}_{\ge 0}} \{c^i d^j \mid \exists i,j \in \mathbb{N}_{\ge 0}: i + j = n \} \\
V &= X^n \cup Y^n \cup S \text{mit n } \in \mathbb{N} \\
\Sigma &= \{ a,b,c,d \} \\
X'^n &= \mid^{n \in \mathbb{N}_{\ge 0}} (\mid^{X \in X^n} X) \\
Y'^n &= \mid^{n \in \mathbb{N}_{\ge 0}} (\mid^{Y \in Y^n} Y) \\
P &= (S , (\mid^{n \in \mathbb{N}_{\ge 0}} X^n Y^n)) \cup X'^n \cup Y'^n \\
G &= (V, \Sigma, P, S)
\end{align*}

\subsection*{G erzeugt L}
Seien $m,n \in \mathbb{N}_{\ge 0}$ beliebig und $p \in \mathbb{N}_{\ge 0}: p \le m+n$ so gilt zu zeigen, dass ein Wort $w \in L$ mit solchen m,n,p und dem sich daraus ergebenden q erzeugt werden kann. Sprich, dass $w \in L(G)$ ist, es gibt also eine Resolution die dieses Wort als Ergebnis hat.
\begin{align*}
S &\rightarrow X^{m+n} Y^{p+q} \\
&\rightarrow a^m b^n c^p d^q
\end{align*}

Dementsprechend ist jedes Wort aus w auch in L(G), also erzeugt G L.
\subsection*{G ist kontextfrei}
zu Zeigen

\section*{5.4}

\subsection*{Präfixtreu}
Es gilt zu zeigen, dass für beliebige $u,v \in \Sigma^*$ f(u) das Präfix von f(uv) ist. Hierzu setzen wir u als $u := a_1,...,a_i$ und v als $v := a_{i+1},...,a_n$. Dann gilt nach Definition 1.62:

\begin{align*}
f(uv) &= \lambda^*(q,uv) \\
&= \lambda^*(q, a_1,...,a_n) \\
&= \lambda(q,a_1) \lambda(\delta(q, a_1), a_2) ... \lambda(\delta(q,a_1...a_{n-1})a_n) \\
&= \lambda(q,a_1) \lambda(\delta(q, a_1), a_2) ... \lambda(\delta(q, a_1 ... a_{i-1}), a_i) \lambda(\delta(q, a_1,..., a_i), a_{i+1} ...  \lambda(\delta(q,a_1...a_{n-1})a_n) \\
&= \lambda^*(q,a_1 ... a_i) \lambda^*(\delta(q, a_1 .... a_i) a_{i+1} ... a_n) \\
&= \lambda^*(q,u) \lambda^*(\delta(q, a_1 .... a_i) v)\\
&= f(u) \lambda^*(\delta(q, a_1 .... a_i) v)
\end{align*}

Damit ist f(u) ein Präfix von f(uv), sprich f ist präfixtreu.

\subsection*{Längenbeschränkt}
Wir zeigen $\forall w \in \Sigma^*: \exists: k \ge 0: \mid f(w) \mid \le k \mid w \mid $ per Induktion über der Länge von w:

\subsubsection*{Induktionsanfang $\mid$ w $\mid$ = 1}
Setze $k := max_{x \in (\Sigma \cup \{ \emptyset \} \cup \{ \epsilon \})}( \mid f(x) \mid)$, so gilt:

\begin{align*}
\mid f(w) \mid &\le k \\
&= k \mid w \mid
\end{align*}

\subsubsection*{Induktionsvoraussetzung $\mid$ w $\mid$ = n}
\begin{align*}
\exists k \ge 0: \mid f(w) \mid \le k \mid w \mid
\end{align*}

\subsubsection*{Induktionsschluss $\mid$ w $\mid$ = n + 1}

Sei w = va, so gilt nach Definition von f:

\begin{align*}
f(va) &= \lambda^*(q_0, va) \\
&= \lambda^*(q_0, v) \lambda(\delta(q_0,v), a) \\
&= f(v)  \lambda(\delta(q_0,v), a)
\end{align*}

Da $\mid f(v) \mid = n$ kann die Induktionsvoraussetzung angewandt werden, also ein k' gesetzt werden für das $\mid f(v) \mid \le k' \mid v \mid$ gilt. Setze nun $k'' := max_{x \in \Sigma}(\mid \lambda(\delta(q_0,v),x) \mid )$ und k := k' + k'', so gilt:

\begin{align*}
\mid f(w) \mid &= \mid f(v) \mid + \mid \lambda(\delta(q_0,v),a) \mid \\
&\le k' \mid v \mid + \mid \lambda(\delta(q_0,v),a) \mid \\
&\le k' \mid v \mid + k'' 1 \\
&= (k' + k'') \cdot (\mid v \mid + 1) \\
&= k \mid w \mid
\end{align*}

\subsection*{Regulär}
zu Zeigen: Ist L regulär, so ist auch f(L) regulär
\end{document}