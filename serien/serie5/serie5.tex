\documentclass[12pt, a4paper]{article}
\usepackage{url,graphicx,tabularx,array,geometry}
\usepackage[utf8]{inputenc}
\usepackage[ngerman]{babel}
\usepackage{paralist}
\usepackage{latexsym}
\usepackage{fancyhdr}
\usepackage{siunitx}
\usepackage{graphicx}
\usepackage{float}
\usepackage{color}
\usepackage{pgf}
\usepackage{tikz}
\usetikzlibrary{arrows,automata}

\pagestyle{fancy}

\usepackage{amsmath}
\usepackage{amsfonts}
\usepackage{amssymb}

\setlength{\parskip}{1ex} %--skip lines between paragraphs
\setlength{\parindent}{0pt} %--don't indent paragraphs

%-- Commands for header
\newcommand{\headerline}{\begin{tabularx}{\textwidth}{X>{\raggedleft}X}\hline\\\end{tabularx}\\[-0.5cm]}
\newcommand{\headerleftright}[2]{\begin{tabularx}{\textwidth}{X>{\raggedleft}X}#1%
& #2\\\end{tabularx}\\[-0.5cm]}
%\linespread{2} %-- Uncomment for Double Space

\usepackage{listings}
\usepackage{color}

\definecolor{dkgreen}{rgb}{0,0.6,0}
\definecolor{gray}{rgb}{0.5,0.5,0.5}
\definecolor{mauve}{rgb}{0.58,0,0.82}

\lstset{frame=tb,
  language=Java,
  aboveskip=3mm,
  belowskip=3mm,
  showstringspaces=false,
  columns=flexible,
  basicstyle={\small\ttfamily},
  numbers=none,
  numberstyle=\tiny\color{gray},
  keywordstyle=\color{blue},
  commentstyle=\color{dkgreen},
  stringstyle=\color{mauve},
  breaklines=true,
  breakatwhitespace=true
  tabsize=3
}

\begin{document}
\renewcommand{\headrulewidth}{0pt}
\fancyhf{}
\fancyhead[L]{
\headerleftright{\textbf{TGI}}{David Elvers, Daniel Schmidt}}
\fancyfoot[C]{\thepage}

\section*{5.2}

\subsection*{G}
Sei G gegeben als $G=(V, \Sigma, P, S)$ mit V = {S,A,B,C,D}, $\Sigma$ = {a,b,c,d} und \\
P = $\{$ (S,A),(S, B), (S, C), (S, D), (S, a), (S, b), (S, c), (S, d), (S, $\varepsilon$), (A, (aSa)), (B, (bSb)), (C, (cSc)), (D, (dSd)) $\}$

\subsection*{Kontextfrei}
Dies gilt, da für alle $(\alpha, \beta) \in P$ gilt, dass $\alpha \in V$ ist. Dementsprechend ist die Grammatik Kontextfrei.

\subsection*{L = L(G)}
Dies zeigen wir, indem wir zeigen, dass alle $w' \in L(G)$ exakt die Bedingung von L erfüllen.  \\
Sei $w \in L$ beliebig, so gilt für die Induktion:

\subsubsection*{ $\mid w \mid = 1$}
In diesem Fall kann für w lediglich gelten $w \in (\{ \emptyset, \varepsilon \} \cup \varSigma )$. Von $w = \emptyset$ oder $w = \varepsilon$ gilt sowohl, dass sie in L sind, als auch das sie in L(G) sind, denn diese sind in jeder Grammatik. (<= ????? Don't know just guessed) Für $w \in \varSigma$ gilt dass diese in L sind, da ein Buchstabe umgedreht weiterhin der selbe ist. Zudem gilt, dass dieser in L(G) ist, da für $(S,X) \in P$ gilt, dass jedes Element aus $\varSigma$ in X ist. Dementsprechend gilt dies für solche w.

\subsubsection*{Induktionsvoraussetzung: $\mid w \mid = n$}
$\forall w \in L(G): w \in L$ (Ist das die richtige Behauptung?)

\subsubsection*{Induktionsschritt: $\mid w \mid = n + 1$}
zu Zeigen

\section*{5.3}
Um zu zeigen, dass eine Sprache kontextfrei ist zeigen wir, dass es eine Grammatik gibt die Kontextfrei ist und diese Sprache erzeugt.
Diese definieren wir wie folgt:

\begin{align*}
X^n &= \bigcup^{n \in \mathbb{N}_{\ge 0}} \{a^i b^j \mid \exists i,j \in \mathbb{N}_{\ge 0}: i + j = n \} \\
Y^n &= \bigcup^{n \in \mathbb{N}_{\ge 0}} \{c^i d^j \mid \exists i,j \in \mathbb{N}_{\ge 0}: i + j = n \} \\
V &= X^n \cup Y^n \cup S \text{mit n } \in \mathbb{N} \\
\Sigma &= \{ a,b,c,d \} \\
X'^n &= \mid^{n \in \mathbb{N}_{\ge 0}} (\mid^{X \in X^n} X) \\
Y'^n &= \mid^{n \in \mathbb{N}_{\ge 0}} (\mid^{Y \in Y^n} Y) \\
P &= (S , (\mid^{n \in \mathbb{N}_{\ge 0}} X^n Y^n)) \cup X'^n \cup Y'^n \\
G &= (V, \Sigma, P, S)
\end{align*}

\subsection*{G erzeugt L}
Seien $m,n \in \mathbb{N}_{\ge 0}$ beliebig und $p \in \mathbb{N}_{\ge 0}: p \le m+n$ so gilt zu zeigen, dass ein Wort $w \in L$ mit solchen m,n,p und dem sich daraus ergebenden q erzeugt werden kann. Sprich, dass $w \in L(G)$ ist, es gibt also eine Resolution die dieses Wort als Ergebnis hat.
\begin{align*}
S &\rightarrow X^{m+n} Y^{p+q} \\
&\rightarrow a^m b^n c^p d^q
\end{align*}

Dementsprechend ist jedes Wort aus w auch in L(G), also erzeugt G L.
\subsection*{G ist kontextfrei}
zu Zeigen

\section*{5.4}


\end{document}