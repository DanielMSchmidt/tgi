\documentclass[12pt, a4paper]{article}
\usepackage{url,graphicx,tabularx,array,geometry}
\usepackage[utf8]{inputenc}
\usepackage[ngerman]{babel}
\usepackage{paralist}
\usepackage{latexsym}
\usepackage{fancyhdr}
\usepackage{siunitx}
\usepackage{graphicx}
\usepackage{float}
\usepackage{color}
\usepackage{pgf}
\usepackage{tikz}
\usetikzlibrary{arrows,automata}

\pagestyle{fancy}

\usepackage{amsmath}
\usepackage{amsfonts}
\usepackage{amssymb}

\setlength{\parskip}{1ex} %--skip lines between paragraphs
\setlength{\parindent}{0pt} %--don't indent paragraphs

%-- Commands for header
\newcommand{\headerline}{\begin{tabularx}{\textwidth}{X>{\raggedleft}X}\hline\\\end{tabularx}\\[-0.5cm]}
\newcommand{\headerleftright}[2]{\begin{tabularx}{\textwidth}{X>{\raggedleft}X}#1%
& #2\\\end{tabularx}\\[-0.5cm]}
%\linespread{2} %-- Uncomment for Double Space

\usepackage{listings}
\usepackage{color}

\definecolor{dkgreen}{rgb}{0,0.6,0}
\definecolor{gray}{rgb}{0.5,0.5,0.5}
\definecolor{mauve}{rgb}{0.58,0,0.82}

\lstset{frame=tb,
  language=Java,
  aboveskip=3mm,
  belowskip=3mm,
  showstringspaces=false,
  columns=flexible,
  basicstyle={\small\ttfamily},
  numbers=none,
  numberstyle=\tiny\color{gray},
  keywordstyle=\color{blue},
  commentstyle=\color{dkgreen},
  stringstyle=\color{mauve},
  breaklines=true,
  breakatwhitespace=true
  tabsize=3
}

\begin{document}
\renewcommand{\headrulewidth}{0pt}
\fancyhf{}
\fancyhead[L]{
\headerleftright{\textbf{TGI}}{David Elvers, Daniel Schmidt}}
\fancyfoot[C]{\thepage}

\section*{5.2}

\subsection*{G}
Sei G gegeben als $G=(V, \varSigma, P, S)$ mit V = {S,A,B,C,D}, $\varSigma$ = {a,b,c,d} und \\ 
P = $\{$ (S,(A $\mid$ B $\mid$ C $\mid$ D $\mid$ a $\mid$ b $\mid$ c $\mid$ d $\mid \varepsilon$)), (A, (aSa)), (B, (bSb)), (C, (cSc)), (D, (dSd)) $\}$

\subsection*{Kontextfrei}
Dies gilt, da für alle $(\alpha, \beta) \in P$ gilt, dass $\alpha \in V$ ist. Dementsprechend ist die Grammatik Kontextfrei.

\subsection*{L = L(G)}
Dies zeigen wir, indem wir zeigen, dass alle $w' \in L(G)$ exakt die Bedingung von L erfüllen.  \\
Sei $w \in L$ beliebig, so gilt für die Induktion:

\subsubsection*{ $\mid w \mid = 1$}
In diesem Fall kann für w lediglich gelten $w \in (\{ \emptyset, \varepsilon \} \cup \varSigma )$. Von $w = \emptyset$ oder $w = \varepsilon$ gilt sowohl, dass sie in L sind, als auch das sie in L(G) sind, denn diese sind in jeder Grammatik. (<= ????? Don't know just guessed) Für $w \in \varSigma$ gilt dass diese in L sind, da ein Buchstabe umgedreht weiterhin der selbe ist. Zudem gilt, dass dieser in L(G) ist, da für $(S,X) \in P$ gilt, dass jedes Element aus $\varSigma$ in X ist. Dementsprechend gilt dies für solche w.

\subsubsection*{Induktionsvoraussetzung: $\mid w \mid = n$}
$\forall w \in L(G): w \in L$ (Ist das die richtige Behauptung?)

\subsubsection*{Induktionsschritt: $\mid w \mid = n + 1$}
zu Zeigen

\section*{5.3}
\section*{5.4}

\end{document}