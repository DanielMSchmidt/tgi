\documentclass[12pt, a4paper]{article}
\usepackage{url,graphicx,tabularx,array,geometry}
\usepackage[utf8]{inputenc}
\usepackage[ngerman]{babel}
\usepackage{paralist}
\usepackage{latexsym}
\usepackage{fancyhdr}
\usepackage{siunitx}
\usepackage{graphicx}
\usepackage{float}
\usepackage{color}
\usepackage{pgf}
\usepackage{tikz}
\usetikzlibrary{arrows,automata}

\pagestyle{fancy}

\usepackage{amsmath}
\usepackage{amsfonts}
\usepackage{amssymb}

\setlength{\parskip}{1ex} %--skip lines between paragraphs
\setlength{\parindent}{0pt} %--don't indent paragraphs

%-- Commands for header
\newcommand{\headerline}{\begin{tabularx}{\textwidth}{X>{\raggedleft}X}\hline\\\end{tabularx}\\[-0.5cm]}
\newcommand{\headerleftright}[2]{\begin{tabularx}{\textwidth}{X>{\raggedleft}X}#1%
& #2\\\end{tabularx}\\[-0.5cm]}
%\linespread{2} %-- Uncomment for Double Space

\usepackage{listings}
\usepackage{color}

\definecolor{dkgreen}{rgb}{0,0.6,0}
\definecolor{gray}{rgb}{0.5,0.5,0.5}
\definecolor{mauve}{rgb}{0.58,0,0.82}

\lstset{frame=tb,
  language=Java,
  aboveskip=3mm,
  belowskip=3mm,
  showstringspaces=false,
  columns=flexible,
  basicstyle={\small\ttfamily},
  numbers=none,
  numberstyle=\tiny\color{gray},
  keywordstyle=\color{blue},
  commentstyle=\color{dkgreen},
  stringstyle=\color{mauve},
  breaklines=true,
  breakatwhitespace=true
  tabsize=3
}

\begin{document}
\renewcommand{\headrulewidth}{0pt}
\fancyhf{}
\fancyhead[L]{
\headerleftright{\textbf{TGI}}{David Elvers, Daniel Schmidt}}
\fancyfoot[C]{\thepage}

\section*{5.2}
Sei G die kontextfreie Sprache von $L = \{w | w^\top = w\}$.
Sei $G=({S, X}, {a,b,c,d}, P, S)$, wobei P definiert ist durch:\\ 
\begin{align*}
S &\rightarrow \epsilon | X\\
X &\rightarrow a|b|c|d|aa|bb|cc|dd|aXa|bXb|cXc|dXd\\
\end{align*}

\subsection*{G erzeugt L}
Das G die Sprache L erzeugt wird wie folgt induktiv gezeigt:

\textbf{Induktions Anfang:}\\
Für $w \in L$ mit $|w| \leq 2$ gilt:
\begin{align*}
\text{F\"ur } \epsilon &\in L \text{ gilt } S \rightsquigarrow X \rightsquigarrow \epsilon \\
\text{F\"ur } a &\in L   \text{ gilt }      S \rightsquigarrow X \rightsquigarrow a\\
\text{F\"ur } b &\in L   \text{ gilt }      S \rightsquigarrow X \rightsquigarrow b\\
\text{F\"ur } c &\in L   \text{ gilt }      S \rightsquigarrow X \rightsquigarrow c\\
\text{F\"ur } d &\in L   \text{ gilt }      S \rightsquigarrow X \rightsquigarrow d\\
\text{F\"ur } aa &\in L   \text{ gilt }      S \rightsquigarrow X \rightsquigarrow aa\\
\text{F\"ur } bb &\in L   \text{ gilt }      S \rightsquigarrow X \rightsquigarrow bb\\
\text{F\"ur } cc &\in L   \text{ gilt }      S \rightsquigarrow X \rightsquigarrow cc\\
\text{F\"ur } dd &\in L   \text{ gilt }      S \rightsquigarrow X \rightsquigarrow dd\\
\end{align*}
\textbf{Induktions Behauptung:}\\
Es gibt ein $w \in L$ das von der Grammatik erzeugt wird.\\
\textbf{Induktions Schluss:}\\
Sei $w \in L$ und von $G$ erzeugbar. Weiter sei $y \in \{a,b,c,d\}$ dann gilt $ywy \in L$ da $(ywy)^\top = ywy$.
F\"ur $ywy \in L$ gilt $S \rightsquigarrow X \rightsquigarrow yXy$. Nach der Induktionsbehauptung kann $w$ durch $G$ erzeugt werden und damit das $X$ zu einem Ausdruck der in $L$ ist aufgel"ost werden.

\section*{5.3}
Um zu zeigen, dass die Sprache L kontextfrei ist müssen wir eine Grammatik G finden, die L erzeugt und kontextfrei ist.

Sei G gegeben als $G = (\{ S,  X_1, X_2, X_{2'}, X_3 \},\{a,b,c,d\}, P, S )$ wobei P definiert ist durch:

\begin{align*}
S &\rightarrow X_1 \\
X_1 &\rightarrow ad \mid a X_1d \mid X_2 \mid X_{2'} \\
X_2 &\rightarrow ac \mid a X_2 c \mid X_3 \\
X_{2'} &\rightarrow bd \mid b X_{2'} d \mid X_3 \\
X_3 &\rightarrow bc \mid b X_3 c \mid \epsilon
\end{align*}

\subsection*{G erzeugt L}
Es gilt zu zeigen, dass jedes Wort in L durch G erzeugt werden kann. Hierzu unterscheiden wir den Fall, dass $m \ge q$ ist und den Fall, dass $m \le q$ ist:
Sei $w \in L$, so gilt $\exists m,n,p,q \in \mathbb{N}: w = a^m b^n c^p d^q$.

\subsubsection*{$m \ge q$}
 Es gilt ebenfalls:
\begin{align*}
&S \vdash_G^{S \rightarrow X_1} \\
&X_1 \vdash_G^{\text{q mal } X_1 \rightarrow a X_1 d} \\
&a^q X_1 d^q \vdash_G^{X_1 \rightarrow X_2} \\
&a^q X_2 d^q \vdash_G^{\text{ m-q mal } X_2 \rightarrow a X_2 c}  \\
&a^q a^{m-q} X_2 c^{m-q} d^q \vdash_G^{X_2 \rightarrow X_3} \\
&a^q a^{m-q} X_3 c^{m-q} d^q \vdash_G^{\text{ n-1 mal } X_3 \rightarrow b X_3 c }\\
&a^q a^{m-q} b^{n-1} X_3 c^{n-1} c^{m-q} d^q \vdash_G^{X_3 \rightarrow bc} \\
&a^q a^{m-q} b^{n-1} bc c^{n-1} c^{m-q} d^q = \\
&a^m b^n c^{n + m - q} d^q =^{\text{Da p = m + n -q}} \\
&a^m b^n c^p d^q \\
\end{align*}

\subsubsection*{$m \leq q$}
\begin{align*}
&S 																			\vdash_G^{S \rightarrow X_1} \\
&X_1 																		\vdash_G^{\text{m mal } X_1 \rightarrow a X_1 d} \\
&a^m X_1 d^m 														\vdash_G^{X_1 \rightarrow X_{2'} } \\
&a^m X_{2'} d^m					 								\vdash_G^{\text{ q-m mal } X_{2'} \rightarrow b X_{2'} d}  \\
&a^m b^{q-m} X_{2'} d^{q-m} d^m 						\vdash_G^{X_{2'} \rightarrow X_3} \\
&a^m b^{q-m} X_3 d^{q-m} d^m							 \vdash_G^{\text{ p-1 mal } X_3 \rightarrow b X_3 c }\\
&a^m b^{q-m} b^{p-1} X_3 c^{p-1} d^{q-m} d^m \vdash_G^{X_3 \rightarrow bc} \\
&a^m b^{q-m} b^{p-1} bc c^{p-1} d^{q-m} d^m	 = \\
&a^m b^{p+q-m}c^{p} d^{q}	 =^{\text{Da n = p+q-m}} \\
&a^m b^n c^p d^q \\
\end{align*}


Hierbei ist zu beachten, dass falls einer der Werte 0 ist direkt zum nächsten Resolutionsschritt übergegangen wird, bzw dass die Gesamtresolution mit einer Resolution nach einem Ergebnis ohne Variable beendet wird.
\subsection*{G ist kontextfrei}
Da alle Regeln aus G die Form haben, dass sie einer Variable eine Transition zuordnen ist diese Grammatik und damit auch die Sprach L kontextfrei.

\section*{5.4}

\subsection*{Präfixtreu}
Es gilt zu zeigen, dass für beliebige $u,v \in \Sigma^*$ f(u) das Präfix von f(uv) ist. Hierzu setzen wir u als $u := a_1,...,a_i$ und v als $v := a_{i+1},...,a_n$. Dann gilt nach Definition 1.62:

\begin{align*}
f(uv) &= \lambda^*(q,uv) \\
&= \lambda^*(q, a_1,...,a_n) \\
&= \lambda(q,a_1) \lambda(\delta(q, a_1), a_2) ... \lambda(\delta(q,a_1...a_{n-1})a_n) \\
&= \lambda(q,a_1) \lambda(\delta(q, a_1), a_2) ... \lambda(\delta(q, a_1 ... a_{i-1}), a_i) \lambda(\delta(q, a_1,..., a_i), a_{i+1} ...  \lambda(\delta(q,a_1...a_{n-1})a_n) \\
&= \lambda^*(q,a_1 ... a_i) \lambda^*(\delta(q, a_1 .... a_i) a_{i+1} ... a_n) \\
&= \lambda^*(q,u) \lambda^*(\delta(q, a_1 .... a_i) v)\\
&= f(u) \lambda^*(\delta(q, a_1 .... a_i) v)
\end{align*}

Damit ist f(u) ein Präfix von f(uv), sprich f ist präfixtreu.

\subsection*{Längenbeschränkt}
Wir zeigen $\forall w \in \Sigma^*: \exists: k \ge 0: \mid f(w) \mid \le k \mid w \mid $ per Induktion über der Länge von w:

\subsubsection*{Induktionsanfang $\mid$ w $\mid$ = 1}
Setze $k := max_{x \in (\Sigma \cup \{ \emptyset \} \cup \{ \epsilon \})}( \mid f(x) \mid)$, so gilt:

\begin{align*}
\mid f(w) \mid &\le k \\
&= k \mid w \mid
\end{align*}

\subsubsection*{Induktionsvoraussetzung $\mid$ w $\mid$ = n}
\begin{align*}
\exists k \ge 0: \mid f(w) \mid \le k \mid w \mid
\end{align*}

\subsubsection*{Induktionsschluss $\mid$ w $\mid$ = n + 1}

Sei w = va, so gilt nach Definition von f:

\begin{align*}
f(va) &= \lambda^*(q_0, va) \\
&= \lambda^*(q_0, v) \lambda(\delta(q_0,v), a) \\
&= f(v)  \lambda(\delta(q_0,v), a)
\end{align*}

Da $\mid f(v) \mid = n$ kann die Induktionsvoraussetzung angewandt werden, also ein k' gesetzt werden für das $\mid f(v) \mid \le k' \mid v \mid$ gilt. Setze nun $k'' := max_{x \in \Sigma}(\mid \lambda(\delta(q_0,v),x) \mid )$ und k := k' + k'', so gilt:

\begin{align*}
\mid f(w) \mid &= \mid f(v) \mid + \mid \lambda(\delta(q_0,v),a) \mid \\
&\le k' \mid v \mid + \mid \lambda(\delta(q_0,v),a) \mid \\
&\le k' \mid v \mid + k'' 1 \\
&= (k' + k'') \cdot (\mid v \mid + 1) \\
&= k \mid w \mid
\end{align*}

\subsection*{Regulär}
Um dies zu zeigen zeigen wir, dass f(L) eine endliche Anzahl an Äquivalenzklassen hat. Hierzu definieren wir die Äquivalenzrelation $\sim$ auf $f(\Sigma^*)\subseteq \tau^*$ definiert als:
\begin{align*}
\sim = \{(a,b) \mid \exists x,y \in \Sigma^*: f(x) = a \wedge f(y) = b \wedge x = y \}
\end{align*}
Es gilt zu zeigen, dass dies eine Äquivalenzrelation ist:

\subsubsection*{Reflexiv}
\begin{align*}
a \sim a &\Longleftrightarrow^{Def. \sim} \exists x,y \in \Sigma^*: f(x) = a \wedge f(y) = a \wedge x = y \\
&\Longleftrightarrow^{Setze y := x} \exists x \in \Sigma^*: f(x) = a \\
&\Longleftrightarrow^{Da a \in f(\Sigma^*)} wahr
\end{align*}

\subsubsection*{Symetrisch}
\begin{align*}
a \sim b &\Longleftrightarrow^{Def. \sim} \exists x,y \in \Sigma^*:  f(x) = a \wedge f(y) = b \wedge x = y \\
&\Longleftrightarrow^{Umbenennung} \exists x,y \in \Sigma^*:  f(x) = b \wedge f(y) = a \wedge x = y \\
&\Longleftrightarrow^{Def. \sim} b \sim a
\end{align*}

\subsubsection*{Transitiv}
\begin{align*}
&a \sim b \wedge b \sim c \\
&\Longleftrightarrow^{Def. \sim} \exists x,y \in \Sigma^*:  f(x) = a \wedge f(y) = b \wedge x = y \wedge \exists x',y' \in \Sigma^*:  f(x') = b \wedge f(y') = c \wedge x' = y' \\ 
&\Longleftrightarrow^{Setze x' := x} \exists x,y,y' \in \Sigma^*:  f(x) = a \wedge f(y) = b \wedge x = y \wedge  f(x) = b \wedge f(y') = c \wedge x = y' \\ 
&\Longleftrightarrow \exists x,y' \in \Sigma^*:  f(x) = a  \wedge f(y') = c \wedge y = y' \\
&\Longleftrightarrow^{Def. \sim} a \sim c
\end{align*}

Somit ist $\sim$ eine Äquivalenzrelation auf $f(\Sigma^*)$. Da f eine Funktion ist und einer Eingabe nur eine Ausgabe zugeordnet ist kann die Anzahl der Äquivalenzklassen von $\sim$ nur kleiner gleich deren von = sein, denn es fallen durch doppelt getroffene Elemente in $f(\Sigma^*)$ höchstens Äquivalenzklassen zusammen, es entstehen aber keine neuen. Da L aber regulär ist gibt es eine endliche Anzahl an Äquivalenzklassen, also hat f(L) auch eine endliche Anzahl an Äquivalenzklassen und ist damit regulär.

\end{document}