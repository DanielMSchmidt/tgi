\documentclass[12pt, a4paper]{article}
\usepackage{url,graphicx,tabularx,array,geometry}
\usepackage[utf8]{inputenc}
\usepackage[ngerman]{babel}
\usepackage{paralist}
\usepackage{latexsym}
\usepackage{fancyhdr}
\usepackage{siunitx}
\usepackage{graphicx}
\usepackage{float}
\usepackage{color}

\pagestyle{fancy}

\usepackage{amsmath}
\usepackage{amsfonts}
\usepackage{amssymb}

\setlength{\parskip}{1ex} %--skip lines between paragraphs
\setlength{\parindent}{0pt} %--don't indent paragraphs

%-- Commands for header
\newcommand{\headerline}{\begin{tabularx}{\textwidth}{X>{\raggedleft}X}\hline\\\end{tabularx}\\[-0.5cm]}
\newcommand{\headerleftright}[2]{\begin{tabularx}{\textwidth}{X>{\raggedleft}X}#1%
& #2\\\end{tabularx}\\[-0.5cm]}
%\linespread{2} %-- Uncomment for Double Space

\usepackage{listings}
\usepackage{color}

\definecolor{dkgreen}{rgb}{0,0.6,0}
\definecolor{gray}{rgb}{0.5,0.5,0.5}
\definecolor{mauve}{rgb}{0.58,0,0.82}

\lstset{frame=tb,
  language=Java,
  aboveskip=3mm,
  belowskip=3mm,
  showstringspaces=false,
  columns=flexible,
  basicstyle={\small\ttfamily},
  numbers=none,
  numberstyle=\tiny\color{gray},
  keywordstyle=\color{blue},
  commentstyle=\color{dkgreen},
  stringstyle=\color{mauve},
  breaklines=true,
  breakatwhitespace=true
  tabsize=3
}

\begin{document}
\renewcommand{\headrulewidth}{0pt}
\fancyhf{}
\fancyhead[L]{
\headerleftright{\textbf{TGI}}{David Elvers, Daniel Schmidt}}
\fancyfoot[C]{\thepage}

\section*{11.2}
\subsection*{1.}
Sei f wie in der Definition zu primär-rekursiv so ergeben sich für g und h
\begin{align*}
g(x) &= c_1^{(1)} \\
h(x,y,z) &= x \cdot z
\end{align*}

Somit gilt $f = PR(c_1^{(0)}, Komp(\cdot, p_1^{(3)}, p_3^{(3)}))$.
\subsection*{2.}
Für diese Aufgabe definieren wir uns die Hilfsfunktionen $Minus: \mathbb{N} \times \mathbb{N} \Rightarrow \mathbb{N}$, welche die zweite Eingabe von der ersten subtrahiert sofern die erste größer als die zweite ist und ansonsten 0 ausgibt und $Decrement: \mathbb{N} \Rightarrow \mathbb{N}$, welches die Eingabe um einen verringert.

Hierbei sei 
\begin{align*}
decrement &= PR(c_0^{(0)}, p_2^{(2)}) \\
minus &= PR(p_1^{(1)}, Komp(decrement, p_1^{(3)}))
\end{align*}

f lässt sich dann wie folgt angeben:

\begin{align*}
f = Komp(+, Komp(minus, p_1^{(2)}, p_2^{(2)}), Komp(minus, p_2^{(2)}, p_1^{(2)}))
\end{align*}

\subsection*{3.}
Sei f hier gegeben als $f = PR(c_0^{(0)}, c_1^{(2)})$

\section*{11.3}
\subsection*{1.}
Sei bininv die Funktion die das binäre Inverse zurück gibt definiert als

\begin{align*}
bininv = PR(c_1^{(0)}, c_0^{(1)})
\end{align*}

Dann ist divides genau das binäre Inverse zu modulo, dementsprechend gilt

\begin{align*}
divides &= komp(bininv, komp(mod, p_1^{(2)}, p_2^{(2)}))
\end{align*}

\subsection*{2.}
Für diese Funktion definieren wir eine Hilfsfunktion $sumdiv: \mathbb{N} \times \mathbb{N} \rightarrow \mathbb{N}$, welche im Grunde $sumdiv(x,y) = \sum_{i = 0}^y divides(i,x)$ ausführt und wie folgt definiert ist

\begin{align*}
sumdiv = PR(c_0^{(1)}, Komp(+, p_3^{(3)}, Komp(divides, p_2^{(3)}, p_1^{(3)})))
\end{align*}

Dann lässt sich prime definieren als

\begin{align*}
prime = komp(bininv, komp(decrement, komp(sumdiv, p_1^{(1)}, komp(decrement, p_1^{(1)}))))
\end{align*}

\section*{11.4}
Um f erfolgreich abzubilden definieren wir uns zuerst die Hilfsfunktion $hoch: \mathbb{N} \times \mathbb{N} \rightarrow \mathbb{N}$, welche eine Basis und einen Exponenten nimmt und $\text{Basis}^{\text{Exponent}}$ rechnet. Diese ist primitiv Rekursiv wie folgt definiert:

\begin{align*}
hoch = PR(c_1^{(1)}, Komp(\cdot, p_1^{(3)}, p_3^{(3)}))
\end{align*}

Dann ist f definiert als

\begin{align*}
f = \mu-OP(Komp(-, p_1^{(1)}, Komp(hoch, c_2^2, p_2^{(2)})))
\end{align*}
\end{document}