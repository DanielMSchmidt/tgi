\documentclass[12pt, a4paper]{article}
\usepackage{url,graphicx,tabularx,array,geometry}
\usepackage[utf8]{inputenc}
\usepackage[ngerman]{babel}
\usepackage{paralist}
\usepackage{latexsym}
\usepackage{fancyhdr}
\usepackage{siunitx}
\usepackage{graphicx}
\usepackage{float}
\usepackage{color}
\usepackage{pgf}
\usepackage{tikz}
\usetikzlibrary{arrows,automata}

\pagestyle{fancy}

\usepackage{amsmath}
\usepackage{amsfonts}
\usepackage{amssymb}
\usepackage{ upgreek }

\setlength{\parskip}{1ex} %--skip lines between paragraphs
\setlength{\parindent}{0pt} %--don't indent paragraphs

%-- Commands for header
\newcommand{\headerline}{\begin{tabularx}{\textwidth}{X>{\raggedleft}X}\hline\\\end{tabularx}\\[-0.5cm]}
\newcommand{\headerleftright}[2]{\begin{tabularx}{\textwidth}{X>{\raggedleft}X}#1%
& #2\\\end{tabularx}\\[-0.5cm]}
%\linespread{2} %-- Uncomment for Double Space

\usepackage{listings}
\usepackage{color}

\definecolor{dkgreen}{rgb}{0,0.6,0}
\definecolor{gray}{rgb}{0.5,0.5,0.5}
\definecolor{mauve}{rgb}{0.58,0,0.82}

\lstset{frame=tb,
  language=Java,
  aboveskip=3mm,
  belowskip=3mm,
  showstringspaces=false,
  columns=flexible,
  basicstyle={\small\ttfamily},
  numbers=none,
  numberstyle=\tiny\color{gray},
  keywordstyle=\color{blue},
  commentstyle=\color{dkgreen},
  stringstyle=\color{mauve},
  breaklines=true,
  breakatwhitespace=true
  tabsize=3
}

\begin{document}
\renewcommand{\headrulewidth}{0pt}
\fancyhf{}
\fancyhead[L]{
\headerleftright{\textbf{TGI}}{David Elvers, Daniel Schmidt}}
\fancyfoot[C]{\thepage}

\section*{8.2}
Sei a = $(Q, \Sigma, \Upgamma, q_0, \delta, F)$ eine Turingmaschine mit $Q = \{ q_0, q_1, q_2, q_3, q_{3'}, q_4, q_5 \}$, $\Sigma = \{ a,b,c \}$, $\Upgamma = \{ a,b,c,a',b',c', \square \}$ und $F = \{ q_4 \}$. Sei zudem $\delta$ bestimmt durch:

\begin{align*}
\delta(q_0, \square) = (\square, r, q_0) &\mid \text{ignoriere Lücken auf der Suche nach a} \\
\delta(q_0, a) = (a', r, q_1) &\mid \text{a wurde gefunden, suche ein b} \\
\delta(q_0, a') = (a', r, q_0) &\mid \text{suche das nächste a} \\
\delta(q_0, b') = (b', l, q_{3'}) &\mid \text{alle a wurden verarbeitet, teste ob das Wort korrekt ist} \\
\delta(q_1, a) = (a, r, q_1) &\mid \text{suche das nächste b} \\
\delta(q_1, a') = (a', r, q_1) &\mid \text{suche das nächste b} \\
\delta(q_1, b) = (b', r, q_2) &\mid \text{b wurde gefunden, suche ein c} \\
\delta(q_2, b) = (b, r, q_2) &\mid \text{suche das nächste c} \\
\delta(q_2, b') = (b', r, q_2) &\mid \text{suche das nächste c} \\
\delta(q_2, c) = (c', r, q_2) &\mid \text{c wurde gefunden, gehe wieder an den Anfang} \\
\forall X \in \Upgamma \backslash \square: \delta(q_3, X) = (X, l, q_3) &\mid \text{gehe an den Anfang} \\
\delta(q_3, \square) = (\square, r, q_0) &\mid \text{Anfang des Wortes gefunden, suche das nächste a} \\
\forall X \in \Upgamma\backslash \square: \delta(q_{3'}, X) = (X, l, q_{3'}) &\mid \text{gehe an den Anfang} \\
\delta(q_{3'}, \square) = (\square, r, q_4) &\mid \text{Anfang des Wortes gefunden, überprüfe die Verarbeitung} \\
\forall X \in \Upgamma \backslash (\square \cup \Sigma): \delta(q_4, X) = (X, r, q_4) &\mid \text{überprüfe ob nur verarbeitete Symbole auftauchen} \\
\delta(q_4, \square) = (\square, r, q_5) &\mid \text{Ende des Wortes und kein Eingabesymbol wurde gefunden} \\
\end{align*}

so gilt $L(a) = \{a^n b^n c^n \mid n \in \mathbb{N} \}$
\section*{8.3}
Sei $\mathfrak{A}$ wie in der Aufgabenstellung gegeben, dann kann man die passende Grammatik $G = (\{S,[q_0Z_0q_0],[q_0Xq_0],[q_0Xq_1]\},\{0,1\},P,S)$ konstruieren. Dabei sei $P$ wie folgt definiert: 
\begin{align*}
S &\rightarrow [q_0Z_0q_0] \big| [q_0Z_0q_1]\\
[q_0Z_0q_0] &\rightarrow 0[q_0Xq_0][q_0Xq_1]\\
[q_0Xq_0] &\rightarrow 0[q_0Xq_0][q_0Xq_1]\\
[q_0Xq_1] &\rightarrow 1\\
\end{align*}

\section*{8.4}
\subsection*{Behauptung}
Der Schnitt einer kontextfreien Sprache mit einer regulären ist kontextfrei.
\subsection*{Beweis}
Sei $L_1$ eine kontextfreie Sprache, so existiert ein Kellerautomat $a_1$, sodass $L(a_1) = L_1$ gilt. Sei $a_1 = (Q_1, \Sigma, \Upgamma, q_{0_1}, Z_0, \Delta_1, F_1)$ \\
Sei $L_2$ eine reguläre Sprache, so existiert ein NEA  $a_{2'}$, sodass $L(a_{2'}) = L_2$ gilt. 
Da man nach Satz 1.19 zu jedem NEA  $a_{2'}$ einen DEA $a_2$ finden kann, sodass gilt $a_{2'} = a_2 \Rightarrow L(a_2) = L_2$. Sei $a_2 = (Q_2, \Sigma, q_{0_2}, \Delta_2, F_2)$ \\
Nach Bemerkung 1.9 ist der Schnitt zweier NEAs der Schnitt beider Sprachen, demnach gilt es zu zeigen, dass $L(a_1 \times a_2)$ kontextfrei ist. 
Nun lässt sich ein Produktautomaten $a_3$ so konstruieren, dass gilt: $a_1 \times a_2 = a_3 = (Q_1 \times Q_2, \Sigma, \Upgamma, (q_{0_1} \times q_{0_2}), Z_0, \Delta, (F_1 \times F_2))$, wobei $\Delta$ definiert, sodass gilt

\begin{align*}
\forall ((q_1, a, \beta), (p_1, \gamma)) \in \Delta_1 \wedge (q_2, a, p_2) \in \Delta_2: (((q_1, q_2), a, \beta), ((p_1, p_2), \gamma)) \in \Delta
\end{align*}

und

\begin{align*}
\forall ((q_1, \epsilon, \beta), (p_1, \gamma)) \in \Delta_1 \wedge (q_2, \delta, q_2) \in \Delta_2: (((q_1, q_2), \epsilon, \beta), ((p_1, q_2), \gamma)) \in \Delta
\end{align*}

Damit ist $a_3$ ein Kellerautomat, also ist $L(a_3)$ kontextfrei, dadurch gilt die Behauptung.
\hfill $\Box$\\
\end{document}
