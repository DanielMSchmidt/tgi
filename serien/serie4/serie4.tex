\documentclass[12pt, a4paper]{article}
\usepackage{url,graphicx,tabularx,array,geometry}
\usepackage[utf8]{inputenc}
\usepackage[ngerman]{babel}
\usepackage{paralist}
\usepackage{latexsym}
\usepackage{fancyhdr}
\usepackage{siunitx}
\usepackage{graphicx}
\usepackage{float}
\usepackage{color}
\usepackage{pgf}
\usepackage{tikz}
\usetikzlibrary{arrows,automata}

\pagestyle{fancy}

\usepackage{amsmath}
\usepackage{amsfonts}
\usepackage{amssymb}

\setlength{\parskip}{1ex} %--skip lines between paragraphs
\setlength{\parindent}{0pt} %--don't indent paragraphs

%-- Commands for header
\newcommand{\headerline}{\begin{tabularx}{\textwidth}{X>{\raggedleft}X}\hline\\\end{tabularx}\\[-0.5cm]}
\newcommand{\headerleftright}[2]{\begin{tabularx}{\textwidth}{X>{\raggedleft}X}#1%
& #2\\\end{tabularx}\\[-0.5cm]}
%\linespread{2} %-- Uncomment for Double Space

\usepackage{listings}
\usepackage{color}

\definecolor{dkgreen}{rgb}{0,0.6,0}
\definecolor{gray}{rgb}{0.5,0.5,0.5}
\definecolor{mauve}{rgb}{0.58,0,0.82}

\lstset{frame=tb,
  language=Java,
  aboveskip=3mm,
  belowskip=3mm,
  showstringspaces=false,
  columns=flexible,
  basicstyle={\small\ttfamily},
  numbers=none,
  numberstyle=\tiny\color{gray},
  keywordstyle=\color{blue},
  commentstyle=\color{dkgreen},
  stringstyle=\color{mauve},
  breaklines=true,
  breakatwhitespace=true
  tabsize=3
}

\begin{document}
\renewcommand{\headrulewidth}{0pt}
\fancyhf{}
\fancyhead[L]{
\headerleftright{\textbf{TGI}}{David Elvers, Daniel Schmidt}}
\fancyfoot[C]{\thepage}

\section*{Aufgabe 4.2}
Sei das Gleichungssystem GS(a) gegeben als: \\

\begin{equation}
\begin{split}
X_0 &= 0X_0 + 1 X_0 + 0X_1 = 0^* 1^* 0 X_1 \\
X_1 &= 0 X_2 \\
X_2 &= 0 X_2 + 1 X_2 + \varepsilon = 0^* 1^* \varepsilon
\end{split}
\end{equation}

Dies lässt sich umformen als

\begin{equation}
\begin{split}
X_0 &= 0^* 1^* 0 X_1 \\
&= 0^* 1^* 0 (0 X_2) \\
&= 0^* 1^* 00(0^* 1^* \varepsilon) \\
&= 0^* 1^* 00 0^* 1^*
\end{split}
\end{equation}

\section*{Aufgabe 4.3}
Wir beweisen dies per Induktion über die Wortlänge:

\subsection*{Induktionsanfang}
Sei $w \in L$ und $\mid w \mid  = 1$, so gilt
\begin{equation}
w	=  a = w^R
\end{equation}

Somit ist $w^R$ regulär.
\subsection*{Induktionsvoraussetzung}
Sei L regulär, $w \in L$ und $\mid w \mid = n$, so ist $w^R$ regulär.

\subsection*{Induktionsschluss}

Sei $\mid w \mid = n + 1$, so gilt:

\begin{equation}
w = a_1 ... a_{n+1}
\end{equation}

Also gilt für $w^R$

\begin{equation}
\begin{split}
w^R &= a_{n+1} a_n ... a_1 \\
&= a_{n+1} + a_n ... a_1
\end{split}
\end{equation}

$a_{n+1}$ ist per Induktionsanfang regulär und $ a_n ... a_1$ per Induktionsvoraussetzung. Dementsprechend sind beide zusammen ebenfalls regulär.

Da alle $w \in L^R$ regulär sind, wenn L regulär ist gilt die Behauptung.

\section*{Aufgabe 4.4}

Sei $A$ DEA der regul\"aren Sprache L.
Und sei $\overline{A}$ der dazu geh\"orige \"Aquivalenzklassen DEA von $A$.\\
Sei $\overline{A}$ wie folgt definiert:
$\overline{A} = ( \overline{Q}, \sum , \overline{q_0}, \overline{\delta}, \overline{F})$

Nach der Definition von MIN(L) darf es folgenden Teil Automaten nicht geben:

\begin{tikzpicture}[->,>=stealth',shorten >=1pt,auto,node distance=2.8cm,
                    semithick]
  
  \node[initial,state] (A)                    {$\overline{q_0}$};
  \node[state]         (B)  [right of=A]      {$\overline{f_0}$};
  \node[state]         (C)  [right of=B]      {$\overline{f_1}$};

  \path (A) edge              node {$u$} (B)
            edge [bend left]  node {$w$} (C)
        (B) edge              node {$v$} (C);
        
\end{tikzpicture}

Mit $\overline{f_0} , \overline{f_1} \in \overline{F}$.\\
$\delta^*(\overline{q_0}, u) = \overline{f_0}$ \hfill da $u \in L$ \\
$\delta^*(\overline{f_0}, v) = \overline{f_1}$ \hfill da $w \in L$ \\

Da es nach Def von MIN keine derartige Zerlegung geben darf gibt es in MIN(L) kein $\overline{f_0}$ derart, wie vorher definiert. Also Fallen Zust"ande des \"Aquivalenzklassen DEA von L weg, wenn man MIN() darauf anwendet. Daher ist auch der Index der \"Aquivalenzklassen endlich und nach dem Satz von Nerode ist MIN(L) regul\"ar. 

\end{document}