\documentclass[12pt, a4paper]{article}
\usepackage{url,graphicx,tabularx,array,geometry}
\usepackage[utf8]{inputenc}
\usepackage[ngerman]{babel}
\usepackage{paralist}
\usepackage{latexsym}
\usepackage{fancyhdr}
\usepackage{siunitx}
\usepackage{graphicx}
\usepackage{float}
\usepackage{color}
\usepackage{pgf}
\usepackage{tikz}
\usetikzlibrary{arrows,automata}

\pagestyle{fancy}

\usepackage{amsmath}
\usepackage{amsfonts}
\usepackage{amssymb}
\usepackage{ upgreek }
\usepackage{listings}


\setlength{\parskip}{1ex} %--skip lines between paragraphs
\setlength{\parindent}{0pt} %--don't indent paragraphs

%-- Commands for header
\newcommand{\headerline}{\begin{tabularx}{\textwidth}{X>{\raggedleft}X}\hline\\\end{tabularx}\\[-0.5cm]}
\newcommand{\headerleftright}[2]{\begin{tabularx}{\textwidth}{X>{\raggedleft}X}#1%
& #2\\\end{tabularx}\\[-0.5cm]}
%\linespread{2} %-- Uncomment for Double Space

\usepackage{listings}
\usepackage{color}

\definecolor{dkgreen}{rgb}{0,0.6,0}
\definecolor{gray}{rgb}{0.5,0.5,0.5}
\definecolor{mauve}{rgb}{0.58,0,0.82}

\lstset{frame=tb,
  language=Java,
  aboveskip=3mm,
  belowskip=3mm,
  showstringspaces=false,
  columns=flexible,
  basicstyle={\small\ttfamily},
  numbers=none,
  numberstyle=\tiny\color{gray},
  keywordstyle=\color{blue},
  commentstyle=\color{dkgreen},
  stringstyle=\color{mauve},
  breaklines=true,
  breakatwhitespace=true
  tabsize=3
}

\begin{document}
\renewcommand{\headrulewidth}{0pt}
\fancyhf{}
\fancyhead[L]{
\headerleftright{\textbf{TGI}}{David Elvers, Daniel Schmidt}}
\fancyfoot[C]{\thepage}

\section*{10.2}
\subsection*{x $\div$ y}
\lstinputlisting{div.loop}

\subsection*{x mod y}
\lstinputlisting{mod.loop}

\section*{10.3}
Sei P das gegebene Programm und beschreibe $P_n$ die nte Programmzeile und $P_{n,m}$ das Teilprogramm von der nten bis zur mten Programmzeile, so gilt: 

Sei k der kleinste Wert für den $pr_1([P_{6,12}]^k (n+1, 0, 1, n, 0)) = 0$ gilt.

\begin{align*}
[P](n, 0, 0, 0, 0) &= [P_{2,14}](n+1, 0, 0, 0, 0)\\ 
&= [P_{3,14}](n+1, 0, 0, 0, 0) \\
&= [P_{4,14}](n+1, 0, 1, 0, 0) \\
&= [P_{5,14}](n+1, 0, 1, n, 0) \\
&= [P_14]([P_{5,12}](n+1, 0, 1, n, 0)) \\
&= [P_14]([P_{6,12}]^k (n+1, 0, 1, n, 0)) \\
\end{align*}

$[P_{6,12}]$ ist hierbei definiert als

\begin{align*}
[P_{6,12}](n, m, o, p, 0) &= [P_{7,12}](n, m+1, o, p, 0) \\
&= [P_{8,12}](n, m+1, o, p, 0) \\
&= [P_{11,12}]([P_9]^{m+1}(n, m+1, o, p, 0)) \\
&= [P_{11,12}](n, m+1, o, p, o \cdot (m+1)) \\
&= [P_{12}](n, m+1, o \cdot (m+1), p, o \cdot (m+1)) \\
&= (n, m+1, o \cdot (m+1), n - o \cdot (m+1), o \cdot (m+1)) \\
\end{align*}

Insgesamt berechnet [P](n) das kleinste m für das gilt $n \le m!$.

\section*{10.4}
Um zu zeigen, dass jede $While$-berechenbare Funktion $While_0$-berechenbar ist, muss f\"ur jede Teilfunktion aus dem $While$ Programm, die es nicht in der $While_0$ definition gibt ein \"aquivalentes Teilprogramm in $While_0$ From gefunden werden. Zu zeigen ist:
\begin{align}
X_i := 0 &\leftrightarrow X_i := 0-1\\
X_i := X_j + X_k &\leftrightarrow X_i := X_j;\ \text{loop} \ X_k \ \text{begin} \ X_i := x_i+1 \ \text{end};\\
X_i := X_j - X_k &\leftrightarrow X_i := X_j;\ \text{loop}\ X_k\ \text{begin}\ X_i := x_i-1\ \text{end};\\
X_i := X_j &\leftrightarrow X_i := x_j+1;\ X_i := X_i-1;\\
\text{loop}\ X_i \ \text{begin}\ Q\ \text{end} &\leftrightarrow X_k := X_i;\ \text{while}\ X_k >  0\ \text{do begin}\ Q;\ X_k:=X_k-1\ \text{end}
\end{align}
Beweis zu 1.:\\
\begin{align*}
x_i:= 0 &\leftrightarrow [P](n_1, n_2, ...)=(n_1, ..., n_{i-1}, 0, n_{i+1}, ...)\\
&\leftrightarrow (n_1,...,n_{i-1}, 0-1,n_{i+1},...)\\
&\leftrightarrow x_i:=0-1
\end{align*}
Beweis zu 4.:\\
Sei $[Q_1]$ die Funktion die die Semantik von $x_i:=x_j +1$ und $[Q_2]$ die Semantik von $X_i:=X_i -1$ beschreibt.
\begin{align*}
X_i:= X_j &\leftrightarrow [P](n_1,n_2,...) = (n_1,...,n_{i-1},n_j,n_{i+1},...)\\
&= (n_1,...,n_{i-1}, n_j+1-1, n_{i+1},...)\\
&\leftrightarrow [Q_2]([Q_1](n_1,n_2,...)\\
&\leftrightarrow X_i:=X_j+1;\ X_i:=X_i-1
\end{align*}
Beweis zu 5.:
\begin{align*}
\text{loop}\ X_i \ \text{begin}\ Q\ \text{end} &\leftrightarrow [P](n_1,n_2,...)=[Z]^n_i(n_1,n2,...) \\
&=([A]([Q](n_1,...)))^n_i\\
&=[Z]^n(n_1,n_2,...)) \ \text{f\"ur das kleinste}\ n\ \text{mit}\ pr_i([Z]^n(n_1,n_2,...))=0
\end{align*}
da $n_i$ mal $n_i-1=0$
\end{document}
