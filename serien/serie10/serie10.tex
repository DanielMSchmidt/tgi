\documentclass[12pt, a4paper]{article}
\usepackage{url,graphicx,tabularx,array,geometry}
\usepackage[utf8]{inputenc}
\usepackage[ngerman]{babel}
\usepackage{paralist}
\usepackage{latexsym}
\usepackage{fancyhdr}
\usepackage{siunitx}
\usepackage{graphicx}
\usepackage{float}
\usepackage{color}
\usepackage{pgf}
\usepackage{tikz}
\usetikzlibrary{arrows,automata}

\pagestyle{fancy}

\usepackage{amsmath}
\usepackage{amsfonts}
\usepackage{amssymb}
\usepackage{ upgreek }
\usepackage{listings}


\setlength{\parskip}{1ex} %--skip lines between paragraphs
\setlength{\parindent}{0pt} %--don't indent paragraphs

%-- Commands for header
\newcommand{\headerline}{\begin{tabularx}{\textwidth}{X>{\raggedleft}X}\hline\\\end{tabularx}\\[-0.5cm]}
\newcommand{\headerleftright}[2]{\begin{tabularx}{\textwidth}{X>{\raggedleft}X}#1%
& #2\\\end{tabularx}\\[-0.5cm]}
%\linespread{2} %-- Uncomment for Double Space

\usepackage{listings}
\usepackage{color}

\definecolor{dkgreen}{rgb}{0,0.6,0}
\definecolor{gray}{rgb}{0.5,0.5,0.5}
\definecolor{mauve}{rgb}{0.58,0,0.82}

\lstset{frame=tb,
  language=Java,
  aboveskip=3mm,
  belowskip=3mm,
  showstringspaces=false,
  columns=flexible,
  basicstyle={\small\ttfamily},
  numbers=none,
  numberstyle=\tiny\color{gray},
  keywordstyle=\color{blue},
  commentstyle=\color{dkgreen},
  stringstyle=\color{mauve},
  breaklines=true,
  breakatwhitespace=true
  tabsize=3
}

\begin{document}
\renewcommand{\headrulewidth}{0pt}
\fancyhf{}
\fancyhead[L]{
\headerleftright{\textbf{TGI}}{David Elvers, Daniel Schmidt}}
\fancyfoot[C]{\thepage}

\section*{10.2}
\subsection*{x $\div$ y}
\lstinputlisting{div.loop}

\subsection*{x mod y}
\lstinputlisting{mod.loop}
\section*{10.3}

Das ganze ist, Ich bekomm X1 und das Ergebnis res ist die kleinste Zahl für die gilt X1 <= res!

Sei P das gegebene Programm und beschreibe $P_n$ die nte Programmzeile und $P_{n,m}$ das Teilprogramm von der nten bis zur mten Programmzeile, so gilt: 

Sei k der kleinste Wert für den $pr_1([P_{6,12}]^k (n+1, 0, 1, n, 0)) = 0$ gilt.

\begin{align*}
[P](n, 0, 0, 0, 0) &= [P_{2,14}](n+1, 0, 0, 0, 0)\\ 
&= [P_{3,14}](n+1, 0, 0, 0, 0) \\
&= [P_{4,14}](n+1, 0, 1, 0, 0) \\
&= [P_{5,14}](n+1, 0, 1, n, 0) \\
&= [P_14]([P_{5,12}](n+1, 0, 1, n, 0)) \\
&= [P_14]([P_{6,12}]^k (n+1, 0, 1, n, 0)) \\
\end{align*}

\section*{10.4}

\end{document}
