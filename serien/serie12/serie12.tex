\documentclass[12pt, a4paper]{article}
\usepackage{url,graphicx,tabularx,array,geometry}
\usepackage[utf8]{inputenc}
\usepackage[ngerman]{babel}
\usepackage{paralist}
\usepackage{latexsym}
\usepackage{fancyhdr}
\usepackage{siunitx}
\usepackage{graphicx}
\usepackage{float}
\usepackage{color}

\pagestyle{fancy}

\usepackage{amsmath}
\usepackage{amsfonts}
\usepackage{amssymb}

\setlength{\parskip}{1ex} %--skip lines between paragraphs
\setlength{\parindent}{0pt} %--don't indent paragraphs

%-- Commands for header
\newcommand{\headerline}{\begin{tabularx}{\textwidth}{X>{\raggedleft}X}\hline\\\end{tabularx}\\[-0.5cm]}
\newcommand{\headerleftright}[2]{\begin{tabularx}{\textwidth}{X>{\raggedleft}X}#1%
& #2\\\end{tabularx}\\[-0.5cm]}
%\linespread{2} %-- Uncomment for Double Space

\usepackage{listings}
\usepackage{color}

\definecolor{dkgreen}{rgb}{0,0.6,0}
\definecolor{gray}{rgb}{0.5,0.5,0.5}
\definecolor{mauve}{rgb}{0.58,0,0.82}

\lstset{frame=tb,
  language=Java,
  aboveskip=3mm,
  belowskip=3mm,
  showstringspaces=false,
  columns=flexible,
  basicstyle={\small\ttfamily},
  numbers=none,
  numberstyle=\tiny\color{gray},
  keywordstyle=\color{blue},
  commentstyle=\color{dkgreen},
  stringstyle=\color{mauve},
  breaklines=true,
  breakatwhitespace=true
  tabsize=3
}

\begin{document}
\renewcommand{\headrulewidth}{0pt}
\fancyhf{}
\fancyhead[L]{
\headerleftright{\textbf{TGI}}{David Elvers, Daniel Schmidt}}
\fancyfoot[C]{\thepage}

\section*{12.2}
\subsection*{1.}
Zu dieser Instanze gibt es für das PCP Problem keine L\"osung, da das Alphabet von A und B nicht übereinstimmt. (in A gibt es kein b)
\subsection*{2.}
L\"osungsmenge: $\{(1,2),(1,2,3,2), ...\}$ also \\
$ba|aa = b|aaa$ bzw.\\
$ba|aa|aaa|aa = b|aaa|aa|aaa$
\subsection*{3.}
L\"osung: $(3,1)$ \\
$abab|a = ab|aba$
\section*{12.3}
\textit{Beweis:} Wir geben eine Transformation $A,B \rightarrow A', B'$ mit
$A,B$ hat spezielle L\"osung $\Leftrightarrow A',B'$ hat L\"osung.\\
Sei smal-PCP das Postsche Korrespondenzproblem mit einem beliebigen zwei Elementigen Alphabet $\Sigma' = \{a,b\}$. Sei weiter zu jedem Element aus dem Alphabet $\Sigma=\{k_1,...,k_n\}$ des normalen PCP das Wort $k_i'=ab^i$ zu geordnet.
F\"ur W\"orter bedeutet das $w=k_1,...k_m \in \Sigma^+$ wird $w'=k_1',...k_m'$ zugeordnet.\\
Daraus folgt, dass $A=\{w_{1a}, ...,w^{pa}\} $ und $ B=\{w_{1b}, ..., w_{qb}\}$ eine L\"osung hat genau dann wenn $A'=\{w_{1a}', ...,w_{ap}'\} $ und $ B'=\{w_{1b}',..., w_{qb}'\}$ eine L\"osung hat.

\section*{12.4}
Da $Konst_1 \in R$ gilt, wobei R die Klasse der Turing-Berechenbaren Funktionen ist, lässt sich der Satz von Rice anwenden, wodurch $C(\{ Konst_1 \}) := \{w \mid \text{die von } M_w \text{ berechnete Funktion liegt in } Konst_1 \}$ unentscheidbar ist. Daraus folgt aber im besonderen auch, dass $Konst_1$ für  $DTM_{\epsilon} $ unentscheidbar ist, daher ist $\Pi$ unentscheidbar.
\end{document}