\documentclass[12pt, a4paper]{article}
\usepackage{url,graphicx,tabularx,array,geometry}
\usepackage[utf8]{inputenc}
\usepackage[ngerman]{babel}
\usepackage{paralist}
\usepackage{latexsym}
\usepackage{fancyhdr}
\usepackage{siunitx}
\usepackage{graphicx}
\usepackage{float}
\usepackage{color}
\usepackage{pgf}
\usepackage{tikz}
\usetikzlibrary{arrows,automata}

\pagestyle{fancy}

\usepackage{amsmath}
\usepackage{amsfonts}
\usepackage{amssymb}

\setlength{\parskip}{1ex} %--skip lines between paragraphs
\setlength{\parindent}{0pt} %--don't indent paragraphs

%-- Commands for header
\newcommand{\headerline}{\begin{tabularx}{\textwidth}{X>{\raggedleft}X}\hline\\\end{tabularx}\\[-0.5cm]}
\newcommand{\headerleftright}[2]{\begin{tabularx}{\textwidth}{X>{\raggedleft}X}#1%
& #2\\\end{tabularx}\\[-0.5cm]}
%\linespread{2} %-- Uncomment for Double Space

\usepackage{listings}
\usepackage{color}

\definecolor{dkgreen}{rgb}{0,0.6,0}
\definecolor{gray}{rgb}{0.5,0.5,0.5}
\definecolor{mauve}{rgb}{0.58,0,0.82}

\lstset{frame=tb,
  language=Java,
  aboveskip=3mm,
  belowskip=3mm,
  showstringspaces=false,
  columns=flexible,
  basicstyle={\small\ttfamily},
  numbers=none,
  numberstyle=\tiny\color{gray},
  keywordstyle=\color{blue},
  commentstyle=\color{dkgreen},
  stringstyle=\color{mauve},
  breaklines=true,
  breakatwhitespace=true
  tabsize=3
}

\begin{document}
\renewcommand{\headrulewidth}{0pt}
\fancyhf{}
\fancyhead[L]{
\headerleftright{\textbf{TGI}}{David Elvers, Daniel Schmidt}}
\fancyfoot[C]{\thepage}

\section*{7.2}
Wir beweisen per Gegenbeweis und nehmen hierzu an, dass $L_1$ kontextfrei ist. \\
So lässt sich das Pumping-Lemma anwenden und es gibt eine kontextfreie Grammatik $G_1$ mit k Variablen und rechter Regelseite der Länge $\le k$ die $L_1$ erzeugt. Sei n zudem die Pumping-konstante und $|z| \ge n$, so gilt:

\begin{align*}
&\exists uvwxy \in \Sigma^*: z = uvwxy \\
\Longleftrightarrow &\exists jklmn \in \mathbb{N}_{\ge 0}: z = a^j a^k  a^l  a^m  a^n \\
\Longleftrightarrow &\exists jklmn \in \mathbb{N}_{\ge 0}: z = a^{j+k+l+m+n}
\end{align*}

Da $z \in L_1$ muss (j+k+l+m+n) prim sein. Nach dem Pumping-Lemma muss auch für ein beliebiges i $u v^i w x^i y \in L_1$ sein. Setze i:= (j+l+n), so muss folgendes $z' = u v^{(j+l+n)} w x^{(j+l+n)} y \in L_1$ gelten. Zudem gilt:

\begin{align*}
z' &= u v^{(j+l+n)} w x^{(j+l+n)} y \\
&= a^j a^{(j+l+n) \cdot k} a^l a^{(j+l+n) \cdot m} a^n \\
&= a^{j + (j+l+n) \cdot k + l + (j+l+n) \cdot m + n} \\
&= a^{(j+l+n) \cdot (k+m+1)} \\
&\Longleftrightarrow (j+l+n) \cdot (k+m+1) \text{ prim} \\
&\Longleftrightarrow^1 \text{WSP}
\end{align*}

1) führt zum Widerspruch, da $vx \neq \epsilon \Longleftrightarrow k+m \ge 1$ gilt.
Somit ist die Sprache nicht kontextfrei.
\section*{7.3}
Wir beweisen per Gegenbeweis und nehmen hierzu an, dass $L_2$ kontextfrei ist. \\
So lässt sich das Pumping-Lemma anwenden und es gibt eine kontextfreie Grammatik $G_2$ mit k Variablen und rechter Regelseite der Länge $\le k$ die $L_2$ erzeugt. Sei n zudem die Pumping-konstante und $|z| \ge n$, so gilt:

\begin{align*}
z &:= uvwxy \\
u &:= a^j \\
v &:= b^k \\
w &:= \epsilon \\
x &:= a^j \\
y &:= b^k \\
j &\neq k 
\end{align*}

So ist $z \in L_2$, dementsprechend müsste nach dem Pumping-Lemma ebenfalls $z' = u v^i w x^i y \in L_2$ sein, jedoch gilt:

\begin{align*}
z' &= u v^i w x^i y \\
&= a^j b^{k \cdot i} \epsilon a^{j \cdot i} b^k \\
\end{align*}

Da  $k \neq j \Rightarrow (k \cdot i) \neq (j \cdot i)$ ist $z' \notin L_2$, dementsprechend ist die Sprache nicht kontextfrei.

\section*{7.4}
Sei $n$ die Pumping Lemma Zahl zu $L$. Jedes Wort $ez \in L$ der L\"ange $\leq n$ l\"asst sich zerlegen in $uvwxy$ mit den Eigenschaften 1,2,3 des Punmping Lemmas. Da $L \subset \{a\}^*$ f\"ur ein Zeichen $a$ gilt, k\"onnen diese Eigenschaften einfacher formuliert werden:
Es gilt $z= a^m = a^ka^l$, wobei $m \geq n$,$k+l = m$,$ 1 \leq l \leq n$, und $a^ka^{il} \in L$ f\"ur $i \in \mathrm{N}$. F\"ur jedes $z \in L$,$|z| \geq n$, insgesamt nur endlich viele $l$-Werte vor, sagen wir $l_1,l_2,...,l_p$. Sei $q \geq n$ eine Zahl, die vor allen $l_i$ geteilt wird (etwa $q = n!$); und sei $q' \geq q$ eine"geeignet gew\"ahlte" Zahl, die wir noch sp\"ater bestimmen. Betrachte die Sprache
\begin{equation*}
L' = \{ x \in L| |x| \le q\} \cup \{a^ra^iq | q \leq r\leq q' , a^r \in L, i \in \mathrm{N} \}
\end{equation*}

Dann ist $L'$ sicherlich regul\"ar, und es ist klar, dass $L' \subset L$ gilt. Wir zeigen wenn $q'$ gen\"ugent gro\ss ist, dann gilt auch $L \subset L'$. Bis zu W\"ortern der L\"ange $\le q$ stimmen $L$ und $L'$ \"uberein. Sei nun $z = a^r$ in $L$ gibt mit $q \leq r \leq q'$ und $r \equiv m (\text{mod}q)$. Damit ist nun alles klar: Wir w\"ahlen $q'$ so gro\ss, dass die W\"orter in $L$ mit den L\"angen $q,...,q'$ alle m\"oglichen Reste modulo $q$ bilden, die unter allen W\"ortern in $L$ (der L\"ange $\geq q$) \"uberhaupt auftreten. Da es nur endlich viele solche Reste gibt, gibt es eine solche endliche Zahl $q'$.  
 
\end{document}
