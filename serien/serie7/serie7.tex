\documentclass[12pt, a4paper]{article}
\usepackage{url,graphicx,tabularx,array,geometry}
\usepackage[utf8]{inputenc}
\usepackage[ngerman]{babel}
\usepackage{paralist}
\usepackage{latexsym}
\usepackage{fancyhdr}
\usepackage{siunitx}
\usepackage{graphicx}
\usepackage{float}
\usepackage{color}
\usepackage{pgf}
\usepackage{tikz}
\usetikzlibrary{arrows,automata}

\pagestyle{fancy}

\usepackage{amsmath}
\usepackage{amsfonts}
\usepackage{amssymb}

\setlength{\parskip}{1ex} %--skip lines between paragraphs
\setlength{\parindent}{0pt} %--don't indent paragraphs

%-- Commands for header
\newcommand{\headerline}{\begin{tabularx}{\textwidth}{X>{\raggedleft}X}\hline\\\end{tabularx}\\[-0.5cm]}
\newcommand{\headerleftright}[2]{\begin{tabularx}{\textwidth}{X>{\raggedleft}X}#1%
& #2\\\end{tabularx}\\[-0.5cm]}
%\linespread{2} %-- Uncomment for Double Space

\usepackage{listings}
\usepackage{color}

\definecolor{dkgreen}{rgb}{0,0.6,0}
\definecolor{gray}{rgb}{0.5,0.5,0.5}
\definecolor{mauve}{rgb}{0.58,0,0.82}

\lstset{frame=tb,
  language=Java,
  aboveskip=3mm,
  belowskip=3mm,
  showstringspaces=false,
  columns=flexible,
  basicstyle={\small\ttfamily},
  numbers=none,
  numberstyle=\tiny\color{gray},
  keywordstyle=\color{blue},
  commentstyle=\color{dkgreen},
  stringstyle=\color{mauve},
  breaklines=true,
  breakatwhitespace=true
  tabsize=3
}

\begin{document}
\renewcommand{\headrulewidth}{0pt}
\fancyhf{}
\fancyhead[L]{
\headerleftright{\textbf{TGI}}{David Elvers, Daniel Schmidt}}
\fancyfoot[C]{\thepage}

\section*{7.2}
Wir beweisen per Gegenbeweis und nehmen hierzu an, dass $L_1$ kontextfrei ist. \\
So lässt sich das Pumping-Lemma anwenden und es gibt eine kontextfreie Grammatik $G_1$ mit k Variablen und rechter Regelseite der Länge $\le k$ die $L_1$ erzeugt.
Zerteilt man ein $z \in L(G_1)$, sodass $z = uvwxy$ gilt und setzt man nun die Variablen $u := w := x := y := \epsilon$, so ist keine der Bedingungen des Pumping-Lemmas verletzt und es gilt:

\begin{equation}
z \in L_1 \Rightarrow z \in L(G_1) \Longleftrightarrow z = uvwxy = v \Longleftrightarrow \exists k \in \mathbb{N}: \text{k ist prim} \wedge v = a^k
\end{equation}

Wenden wir nun das Pumping-Lemma an, so muss ebenfalls gelten

\begin{equation}
v^2 \in L(G_1) \Longleftrightarrow \exists k': \text{k' ist prim} \wedge v^2 = a^{k'}
\end{equation}

Da dieses $k' = k \cdot 2$ sein müsste ist dies keine Primzahl mehr, es gibt also einen Widerspruch

\section*{7.3}
Wir beweisen per Gegenbeweis und nehmen hierzu an, dass $L_2$ kontextfrei ist. \\
So lässt sich das Pumping-Lemma anwenden und es gibt eine kontextfreie Grammatik $G_2$ mit k Variablen und rechter Regelseite der Länge $\le k$ die $L_2$ erzeugt.
Zerteilt man ein $z \in L(G_2)$, sodass $z = uvwxy$ gilt und setzt man nun die Variablen $u := y := \epsilon$, $v := ab$, $w := a$ und $x := b$, so ist keine der Bedingungen des Pumping-Lemmas verletzt und es gilt:


\begin{equation}
v^2 w x^2 = abab a bb \notin L(G_2)  \Longleftrightarrow v^2 w x^2 \notin L_2
\end{equation}

Dies ist ein Widerspruch zum Pumping-Lemma.

\section*{7.4}

\end{document}
