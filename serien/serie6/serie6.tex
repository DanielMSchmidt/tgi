\documentclass[12pt, a4paper]{article}
\usepackage{url,graphicx,tabularx,array,geometry}
\usepackage[utf8]{inputenc}
\usepackage[ngerman]{babel}
\usepackage{paralist}
\usepackage{latexsym}
\usepackage{fancyhdr}
\usepackage{siunitx}
\usepackage{graphicx}
\usepackage{float}
\usepackage{color}
\usepackage{pgf}
\usepackage{tikz}
\usetikzlibrary{arrows,automata}

\pagestyle{fancy}

\usepackage{amsmath}
\usepackage{amsfonts}
\usepackage{amssymb}

\setlength{\parskip}{1ex} %--skip lines between paragraphs
\setlength{\parindent}{0pt} %--don't indent paragraphs

%-- Commands for header
\newcommand{\headerline}{\begin{tabularx}{\textwidth}{X>{\raggedleft}X}\hline\\\end{tabularx}\\[-0.5cm]}
\newcommand{\headerleftright}[2]{\begin{tabularx}{\textwidth}{X>{\raggedleft}X}#1%
& #2\\\end{tabularx}\\[-0.5cm]}
%\linespread{2} %-- Uncomment for Double Space

\usepackage{listings}
\usepackage{color}

\definecolor{dkgreen}{rgb}{0,0.6,0}
\definecolor{gray}{rgb}{0.5,0.5,0.5}
\definecolor{mauve}{rgb}{0.58,0,0.82}

\lstset{frame=tb,
  language=Java,
  aboveskip=3mm,
  belowskip=3mm,
  showstringspaces=false,
  columns=flexible,
  basicstyle={\small\ttfamily},
  numbers=none,
  numberstyle=\tiny\color{gray},
  keywordstyle=\color{blue},
  commentstyle=\color{dkgreen},
  stringstyle=\color{mauve},
  breaklines=true,
  breakatwhitespace=true
  tabsize=3
}

\begin{document}
\renewcommand{\headrulewidth}{0pt}
\fancyhf{}
\fancyhead[L]{
\headerleftright{\textbf{TGI}}{David Elvers, Daniel Schmidt}}
\fancyfoot[C]{\thepage}


\section*{6.2)}
\textbf{Akzeptierte Ausdr\"ucke:}
\begin{enumerate}[]
\item $ba(a^*b^*)^*$
\item $bb(a^*b^*)^*$
\item $aaa(a^*b^*)^*$
\item $aab(a^*b^*)^*$
\item $aba(a^*b^*)^*$
\item $abb(a^*b^*)^*$
\end{enumerate}
\textbf{Daraus ableitbare Grammatik:}
\begin{align*}
S &\rightarrow X \\
X &\rightarrow aY|bQ \\
Y &\rightarrow aQ|bQ \\
Q &\rightarrow a|b|aQ|bQ \\
\end{align*}


\section*{6.3)}
\subsection*{Schritt 1:} 
\begin{align*}
S &\rightarrow aAA|BbB \\
A &\rightarrow Bb|Ba|a \\
B &\rightarrow bBC|bAC \\
C &\rightarrow Bb|a \\
\end{align*}

\subsection*{Schritt 2:}
\begin{align*}
S &\rightarrow X_aAA|BX_bB \\
A &\rightarrow BX_b|BX_a|a \\
B &\rightarrow X_bBC|X_bAC \\
C &\rightarrow BX_b | a \\
X_a &\rightarrow a \\
X_b &\rightarrow b \\
\end{align*}

\subsection*{Schritt 3:}
\begin{align*}
S &\rightarrow X_aA_1|BB_1 \\
A_1 &\rightarrow AA \\
B_1 &\rightarrow X_bB \\
A &\rightarrow BX_b | BX_a | a \\
B &\rightarrow X_bB_2|X_bA_2 \\
B_2 &\rightarrow BC \\
A_2 &\rightarrow AC \\
C &\rightarrow BX_b|a \\
X_a &\rightarrow a \\
X_b &\rightarrow b \\
\end{align*}

\section*{6.4)}

Die durch die gegeben Grammatik erzeugte Sprache ist definiert durch:

\begin{equation}
L := \{ w \in \Sigma^* \mid \text{   } \mid w \mid_{a} = \mid w \mid_{b} \wedge \forall u,v \in \Sigma^*: u \cdot v = w \Rightarrow \mid u \mid_{a} \ge \mid u \mid_{b} \}
\end{equation}

Nun gilt es zu zeigen, dass L = L(G) ist. Dies zeigen wir per Induktion:

\subsection*{$w \in L \Rightarrow w \in L(G)$}
\subsubsection*{Induktionsanfang}
Sei $\mid w \mid = 0$, so folgt daraus, dass $w = \epsilon$ sein muss. Dementsprechend kann w von S aus resolviert werden, ist also in L.\\
Für w gilt ebenfalls $w = 0 = \mid w \mid_a = \mid w \mid_b $. Da es lediglich ein u und v gibt, sodass $ \epsilon = uv$ gilt und zwar $\epsilon := u := v$.

\subsubsection*{Induktionsvoraussetzung}
$ w \in L \wedge \mid w \mid = n  \Rightarrow w \in L(G)$

\subsubsection*{Induktionsschluss}
Sei $n \in \mathbb{N}$ und $\mid w \mid = n +2 \ge 2$. Dann ist w = aubv mit $u,v \in L$, da w mit einem a beginnen muss, um die Bedingung $\forall u,v \in \Sigma^*: uv = w \Rightarrow \mid u \mid_a \ge \mid u \mid_b$ zu erfüllen. Dementsprechend muss es aber Ableitungen geben, sodass gilt: 

\begin{align*}
S &\vdash^* aSbS \\
&= aubv \\
&= w
\end{align*}

\subsection*{$w \in L(G) \Rightarrow w \in L$}
\subsubsection*{Induktionsanfang}
Sei $\mid w \mid = 0$, so folgt daraus, dass $w = \epsilon$ sein muss. Dementsprechend kann w von S aus resolviert werden, ist also in L.\\
Für w gilt ebenfalls $w = 0 = \mid w \mid_a = \mid w \mid_b $. Da es lediglich ein u und v gibt, sodass $ \epsilon = uv$ gilt und zwar $\epsilon := u := v$.

\subsubsection*{Induktionsvoraussetzung}
$w \in L(G) \wedge \mid w \mid = n  \Rightarrow w \in L$

\subsubsection*{Induktionsschluss}
Sei w der Länge n+2, so muss es im Vergleich zu einem w' der Länge n eine weitere Resolution der Form 

\begin{align*}
S \vdash aSbS
\end{align*}

geben. Dementsprechend muss es in w Teilwörter geben, sodass w = aubv gilt.
Das beide Resolvierbar sind wissen wir da $\mid u \mid \le n \wedge \mid v \mid \le n \wedge \mid uv \mid = n$ gilt. Daher gilt auch $\mid u \mid_a = \mid u \mid_b$ und $\mid v \mid_a = \mid v \mid_b$. Also gilt $\mid w  \mid_a = 1 + \mid uv \mid_a = 1 + \mid uv \mid_b = \mid w \mid_b$. Da a vor b angefügt wurde gilt ebenfalls 

\begin{equation}
\forall x,y \in \Sigma^*: x \cdot y = w \Rightarrow \mid x \mid_{a} \ge \mid y \mid_{b}
\end{equation}

Dementsprechend ist $w \in L$.
\end{document}
