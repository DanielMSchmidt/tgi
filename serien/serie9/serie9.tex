\documentclass[12pt, a4paper]{article}
\usepackage{url,graphicx,tabularx,array,geometry}
\usepackage[utf8]{inputenc}
\usepackage[ngerman]{babel}
\usepackage{paralist}
\usepackage{latexsym}
\usepackage{fancyhdr}
\usepackage{siunitx}
\usepackage{graphicx}
\usepackage{float}
\usepackage{color}
\usepackage{pgf}
\usepackage{tikz}
\usetikzlibrary{arrows,automata}

\pagestyle{fancy}

\usepackage{amsmath}
\usepackage{amsfonts}
\usepackage{amssymb}
\usepackage{ upgreek }

\setlength{\parskip}{1ex} %--skip lines between paragraphs
\setlength{\parindent}{0pt} %--don't indent paragraphs

%-- Commands for header
\newcommand{\headerline}{\begin{tabularx}{\textwidth}{X>{\raggedleft}X}\hline\\\end{tabularx}\\[-0.5cm]}
\newcommand{\headerleftright}[2]{\begin{tabularx}{\textwidth}{X>{\raggedleft}X}#1%
& #2\\\end{tabularx}\\[-0.5cm]}
%\linespread{2} %-- Uncomment for Double Space

\usepackage{listings}
\usepackage{color}

\definecolor{dkgreen}{rgb}{0,0.6,0}
\definecolor{gray}{rgb}{0.5,0.5,0.5}
\definecolor{mauve}{rgb}{0.58,0,0.82}

\lstset{frame=tb,
  language=Java,
  aboveskip=3mm,
  belowskip=3mm,
  showstringspaces=false,
  columns=flexible,
  basicstyle={\small\ttfamily},
  numbers=none,
  numberstyle=\tiny\color{gray},
  keywordstyle=\color{blue},
  commentstyle=\color{dkgreen},
  stringstyle=\color{mauve},
  breaklines=true,
  breakatwhitespace=true
  tabsize=3
}

\begin{document}
\renewcommand{\headrulewidth}{0pt}
\fancyhf{}
\fancyhead[L]{
\headerleftright{\textbf{TGI}}{David Elvers, Daniel Schmidt}}
\fancyfoot[C]{\thepage}

\section*{9.2}

$\mathfrak{A} = (\{q_0,q_1, q_f\}, \{1,0\},\{0,1, \Box \}, \delta, q_0, \Box, \{q_f\})$\\
Dabei ist $\delta$ wie folgt definiert:

\begin{align*}
\delta(q_0,0) = (q_0,0,r) \hfill &\text{Suche das LSB (least significant bit)}\\
\delta(q_0,1) = (q_0,1,r) \hfill &\text{Suche das LSB (least significant bit)}\\
\delta(q_0, \Box) =(q_1,\Box,l) \hfill &\text{Found the LSB (least significant bit)}\\
\delta(q_1, 0) =(q_f,1,l) \hfill &\text{Wenn das LSB 0 ist addiere einen drauf und wir sind fertig}\\
\delta(q_1,1) = (q_1, 0, l) \hfill &\text{Wenn das LSB 1 ist setze es auf 0 und addiere es auf das nächste Element}\\
\delta(q_1, \Box) = (q_f,1,n) \hfill &\text{Wenn das Wort zu kurz ist setze eine 1 vor das MSB (most significant bit)}\\
\end{align*}
Dabei dient $q_0$ zum bewegen des Lese/Schreibkopfes nach ganz rechts zum kleinsten Bit und $q_1$ addiert 1 und ändert alle Bits von rechts nach links.

\section*{9.3}
\begin{align*}
\delta(q_0,\{0,0\}) &= (q_0,\{1,0\},rr)\\
\delta(q_0,\{0,1\}) &= (q_0,\{1,1\},rr)\\
\delta(q_0,\{0,\Box\}) &= (q_0,\{1,\Box\},rn)\\
\delta(q_0,\{1,0\}) &= (q_0,\{0,0\},rr)\\
\delta(q_0,\{1,1\}) &= (q_0,\{0,1\},rr)\\
\delta(q_0,\{1,\Box\}) &= (q_0,\{0,\Box\},rn)\\
\delta(q_0,\{\Box,0\}) &= (q_0,\{\Box,0\},nr)\\
\delta(q_0,\{\Box,1\}) &= (q_0,\{\Box,1\},nr)\\
\delta(q_0,\{\Box,\Box\}) &= (q_1,\{\Box,\Box\},ll)\\
\\
\delta(q_1,\{0, \Box\}) &= (q_2, \{1,\Box\}, ln)\\
\delta(q_1,\{1, \Box\}) &= (q_1, \{0,\Box\}, ln)\\
\delta(q_1,\{\Box, \Box\}) &= (q_2, \{1,\Box\}, ln)\\
\\
\delta(q_2,\{0, \Box\}) &= (q_2, \{0, \Box\}, rn)\\
\delta(q_2, \{1, \Box\}) &= (q_2, \{1, \Box\}, rn)\\
\delta(q_2, \{\Box, \Box\}) &= (q_3, \{\Box, \Box\}, nn)\\
\\
\delta(q_3, \{0,0\}) &=(q_3, \{0,\Box\},ll)\\
\delta(q_3, \{0,1\}) &=(q_3, \{1,\Box\},ll)\\
\delta(q_3, \{1,0\}) &=(q_3, \{1,\Box\},ll)\\
\delta(q_3, \{1,1\}) &=(q_4, \{0,\Box\},ll)\\
\delta(q_3, \{\Box,0\}) &=(q_3, \{0,\Box\},ll)\\
\delta(q_3, \{\Box,1\}) &=(q_3, \{1,\Box\},ll)\\
\delta(q_3, \{1,\Box\}) &=(q_3, \{1,\Box\},ll)\\
\delta(q_3, \{0,\Box\}) &=(q_3, \{0,\Box\},ll)\\
\delta(q_3, \{\Box,\Box\}) &=(q_f, \{\Box,\Box\},ll)\\
\\
\delta(q_4, \{0,0\}) &=(q_3, \{0,\Box\},ll)\\
\delta(q_4, \{0,1\}) &=(q_4, \{0,\Box\},ll)\\
\delta(q_4, \{1,0\}) &=(q_4, \{0,\Box\},ll)\\
\delta(q_4, \{1,1\}) &=(q_4, \{1,\Box\},ll)\\
\delta(q_4, \{\Box,0\}) &=(q_3, \{1,\Box\},ll)\\
\delta(q_4, \{\Box,1\}) &=(q_4, \{0,\Box\},ll)\\
\delta(q_4, \{0,\Box\}) &=(q_3, \{1,\Box\},ll)\\
\delta(q_4, \{1,\Box\}) &=(q_4, \{0,\Box\},ll)\\
\delta(q_4, \{\Box,\Box\}) &=(q_f, \{\Box,\Box\},ll)\\
\end{align*}

In $q_0$ wandern beide Lese-/Schreibköpfe zum kleinsten Bit beider Zahlen und invertiert dabei die Zahl auf Band 1.\\
In $q_1$ wird 1 auf die Zahl auf Band 1 addiert.
In $q_2$ werden der Lese und Schreibkopf wieder zum kleinsten Bit der ersten Zahl bewegt.
In $q_3$ werden beide Zahlen Bitweise addiert ohne Übertrag.
In $q_4$ werden beide Zahlen unter Berücksichtigung eines Übertragsbit bitweise addiert.


\section*{9.4}
Sei L eine deterministisch kontextfreie Sprache so gibt es einen DPDA $a = (Q, \Sigma, \Upgamma, q_0, Z_0, \Delta, F)$ für den gilt $L = L(a)$.
Um zu zeigen, dass Min(L) deterministisch kontextfrei ist gilt es zu zeigen, dass ein DPDA a' existiert für welchen Min(L) = L(a') gilt. \\

Sei $a' = (Q, \Sigma, \Upgamma, q_0, Z_0, \Delta', F)$ mit $\Delta'$ definiert als: 

\begin{align*}
\Delta' &= \{(q, a, \gamma, \gamma*, q') \mid  (q, a, \gamma, \gamma*, q') \in \Delta \wedge q \notin F \}
\end{align*}

Somit gilt für jedes von a' erkannte Wort u, dass es kein Wort gibt, welches u als Präfix hat. Dies ist eben die Bedingung der minimalen Sprache, dementsprechend gilt L(a') = Min(L), daher ist Min(L) kontextfrei.


\end{document}
