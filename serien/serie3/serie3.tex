\documentclass[12pt, a4paper]{article}
\usepackage{url,graphicx,tabularx,array,geometry}
\usepackage[utf8]{inputenc}
\usepackage[ngerman]{babel}
\usepackage{paralist}
\usepackage{latexsym}
\usepackage{fancyhdr}
\usepackage{siunitx}
\usepackage{graphicx}
\usepackage{float}
\usepackage{color}

\pagestyle{fancy}

\usepackage{amsmath}
\usepackage{amsfonts}
\usepackage{amssymb}

\setlength{\parskip}{1ex} %--skip lines between paragraphs
\setlength{\parindent}{0pt} %--don't indent paragraphs

%-- Commands for header
\newcommand{\headerline}{\begin{tabularx}{\textwidth}{X>{\raggedleft}X}\hline\\\end{tabularx}\\[-0.5cm]}
\newcommand{\headerleftright}[2]{\begin{tabularx}{\textwidth}{X>{\raggedleft}X}#1%
& #2\\\end{tabularx}\\[-0.5cm]}
%\linespread{2} %-- Uncomment for Double Space

\usepackage{listings}
\usepackage{color}

\definecolor{dkgreen}{rgb}{0,0.6,0}
\definecolor{gray}{rgb}{0.5,0.5,0.5}
\definecolor{mauve}{rgb}{0.58,0,0.82}

\lstset{frame=tb,
  language=Java,
  aboveskip=3mm,
  belowskip=3mm,
  showstringspaces=false,
  columns=flexible,
  basicstyle={\small\ttfamily},
  numbers=none,
  numberstyle=\tiny\color{gray},
  keywordstyle=\color{blue},
  commentstyle=\color{dkgreen},
  stringstyle=\color{mauve},
  breaklines=true,
  breakatwhitespace=true
  tabsize=3
}

\begin{document}
\renewcommand{\headrulewidth}{0pt}
\fancyhf{}
\fancyhead[L]{
\headerleftright{\textbf{TGI - Serie 3}}{David Elvers, Daniel Schmidt}}
\fancyfoot[C]{\thepage}

\section*{3.3 Pumping - Lemma}

Wir beweisen durch Widerspruch und nehmen an, dass L regulär ist. Somit gibt es ein $n \in \mathbb{N}$ für das alle Wörter $x \in L$ die Länge n haben. Da in L nur Wörter deren Länge Prim ist sind, sei $x := a^n$ und betrachtet man eine beliebige Zerlegung x = uvw und setze  s := 2n  , so gilt:
\begin{equation}
\begin{split}
\exists j,k,l: 0 \le j,k,l \le n: x &= uvw \\
&= a^j a^k a^l \\
&=^{Pumping - Lemma }a^j a^{s - (j + k)} a^k \\
&= a^s \\
&= a^{2n} \\
&\Rightarrow \mid a^{2n} \mid = 2n
\end{split}
\end{equation}

Damit ist die Länge des Wortes nicht mehr Prim, also ist das Wort nicht in der Sprache und damit L nicht regulär.
 
\end{document}

